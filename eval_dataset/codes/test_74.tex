\documentclass[12pt,reqno]{article}
\usepackage{amsmath}
\usepackage{amssymb}
\usepackage{graphics,color}
\usepackage{tikz}
\usetikzlibrary{calc}

\begin{document}

\begin{tikzpicture}[scale=1.0] % Scale the entire drawing
    % Define dimensions
    \def\stripWidth{1.5} % Width of the strip
    \def\stripHeight{3.75} % Height of the strip
    \def\semiCircleRadius{\stripWidth / 2} % Radius of the semi-circle
    \def\foldDepth{1.4} % Depth of the fold, horizontal displacement

    % Function to draw the strip with half-disk
    \newcommand{\drawStripWithHalfDisk}[2]{
        % Draw the vertical rectangular strip without the bottom edge
        \draw (#1, #2 + \stripHeight) -- (#1, #2); % Left edge
        \draw (#1 + \stripWidth, #2) -- (#1 + \stripWidth, #2 + \stripHeight); % Right edge
        \draw (#1, #2 + \stripHeight) -- (#1 + \stripWidth, #2 + \stripHeight); % Top edge

        % Draw the half-disk at the bottom of the strip
        \draw (#1, #2) arc (180:360:\semiCircleRadius);

        % Add a dot at the geometric center of the semi-circle
        \filldraw [black] (#1 + \semiCircleRadius, #2) circle (2pt);
    }

    % Function to draw the folded strip
    \newcommand{\drawFoldedStrip}[2]{
        % Draw the main vertical rectangular part of the strip
        \draw (#1, #2) rectangle (#1 + \stripWidth/2, #2 + \stripHeight);

        % Coordinates for parallelogram (fold)
        \coordinate (A) at (#1, #2);
        \coordinate (B) at (#1 + \stripWidth/2, #2 + \stripHeight);
        \coordinate (C) at (#1 + \stripWidth/2 - \foldDepth/2, #2 + \stripHeight - \foldDepth/10); 
        \coordinate (D) at (#1, #2); 

        % Draw the parallelogram (fold)
        \draw (B) -- (C) -- (D) -- (A);
    }

    % Draw the folded strip
    \drawFoldedStrip{0}{0}

    % Draw the strip with half-disk to the right of the folded strip
    \drawStripWithHalfDisk{2.5}{0} % Adjusted position

\end{tikzpicture}

\end{document}