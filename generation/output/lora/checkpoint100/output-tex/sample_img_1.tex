\begin{tikzpicture}[baseline=(current  bounding  box.center)]
\begin{axis}[
    xmin=0,
    xmax=10,
    ymin=0,
    ymax=5,
    axis x line=bottom,
    axis y line=left,
    xtick=\empty,
    ytick=\empty,
    xlabel={Input (capital per worker)},
    ylabel={Output\\per worker},
    legend pos=north west,
    clip=false,
    scale=2
]

% draw the two curves
\addplot [
    domain=0:8, 
    samples=100, 
    color=black!50
]
{(x^2)/3};

\addplot [
    domain=0:10, 
    samples=100, 
    color=black!70
]
{((x+4)^2)/9};

% draw the line where the functions intersect
\addplot [black] coordinates {(0,0)(8,8)};

% draw the label for each function
\node [anchor=south east] at (axis cs:6.2, 3.5) {$y=G(N')f(k)$};
\node [anchor=south east] at (axis cs:6.2, 2.5) {$y=G(N)f(k)$};

\end{axis}
\end{tikzpicture}