\documentclass{article}
\usepackage{amsmath,amssymb}
\usepackage{tikz}
\usetikzlibrary{arrows.meta,automata}
\tikzset{
    ->, % makes the edges directed
    >=latex', % makes the arrow heads bold
    node distance=2cm, % specifies the minimum distance between two nodes. Change if necessary.
    thick,% defines line thickness
    font=\small % sets the default font
}

\begin{document}

\begin{figure}[htbp]
\begin{center}
\begin{tikzpicture}[every node/.style={circle,draw,fill=white,inner sep=5pt}]
    \node (type) at (0,0) {Type};
    \node (size) at (-3,-2) {Size};
    \node (weight) at (-1,-2) {Weight};
    \node (texture) at (1,-2) {Texture};
    \node (rigidity) at (3,-2) {Rigidity};
    \node (gripper) at (-1,-4) {Gripper};
    \node (robot) at (-3,-6) {Robot};
    \node (target) at (1,-4) {Target};
    \node (goal) at (3,-6) {Goal};
    \node (success) at (3,-4) {Success};
    \path [->]
        (type) edge (size)
        (type) edge (weight)
        (type) edge (texture)
        (type) edge (rigidity)
        (size) edge (gripper)
        (weight) edge (gripper)
        (texture) edge (gripper)
        (rigidity) edge (gripper)
        (gripper) edge (robot)
        (gripper) edge (target)
        (robot) edge (target)
        (goal) edge (target)
        (target) edge (success);
\end{tikzpicture}
\end{center}
\caption{An example of the relationships between the operator-identified variables for a pick-and-place task. Each edge corresponds to a direct causal relationship between the connected variables. Changes in the source of an edge have a direct consequence on the target of an edge.}
\label{fig:type}
\end{figure}

\end{document}