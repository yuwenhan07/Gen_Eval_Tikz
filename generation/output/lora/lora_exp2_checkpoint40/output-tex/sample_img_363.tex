documentclass[]
\usepackage{tikz}
\usetikzlibrary{matrix}

\begin{document}
\begin{figure}[ht]
    \centering
    \subfloat[][]{
        \begin{tikzpicture}[>=latex,
            every node/.style={minimum width=1cm, minimum height=1cm, inner sep=0pt},
            ]
            \matrix[matrix of math nodes,nodes={draw},column sep=0pt,row sep=0pt]{%
                & x_1 & y \\
                x_1 & w_h & w_l \\
                y & w_h & w_l \\
            };
            \draw (row 2-|column 1.east) |- (row 1-|column 3.west);
        \end{tikzpicture}
        \label{fig:transition_matrix}
    }
    \hfill
    \subfloat[][]{
        \begin{tikzpicture}[>=latex,
            every node/.style={minimum width=1cm, minimum height=1cm, inner sep=0pt},
            ]
            \matrix[matrix of math nodes,nodes={draw},column sep=0pt,row sep=0pt]{%
                & x_1 & x_2 & y \\
                x_1 & w_l & w_h & w_l \\
                x_2 & w_h & w_l & w_l \\
                y & w_h & w_l & w_l \\
            };
            \draw (row 2-|column 1.east) |- (row 1-|column 3.west);
        \end{tikzpicture}
        \label{fig:edge_transition_matrix}
    }
    \hfill
    \subfloat[][]{
        \begin{tikzpicture}[>=latex,
            every node/.style={minimum width=1cm, minimum height=1cm, inner sep=0pt},
            ]
            \matrix[matrix of math nodes,nodes={draw},column sep=0pt,row sep=0pt]{%
                & x_1 & x_2 & x_3 & y \\
                x_1,x_1 & w_l & w_h & w_l & w_l \\
                x_1,x_2 & w_l & w_l & w_h & w_l \\
                x_1,x_3 & w_h & w_l & w_l & w_l \\
                x_1,y & w_h & w_l & w_l & w_l \\
            };
            \draw (row 2-|column 1.east) |- (row 1-|column 4.west);
        \end{tikzpicture}
        \label{fig:path_transition_matrix}
    }
\end{figure}
\end{document}