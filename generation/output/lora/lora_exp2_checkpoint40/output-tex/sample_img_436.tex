documentclass{article}
\usepackage{tikz}

\begin{document}
\begin{figure}
    \centering
    \begin{tikzpicture}[scale=0.5]
        % draw nodes
        \foreach \i in {1,...,6} {
            \pgfmathtruncatemacro{\result}{mod(\i+3,2)}
            \ifnum\result=1\relax
                \node[circle, fill=black, scale=0.25] (\i) at (-2*\i+4, 0) {};
            \else
                \node[circle, fill=black, scale=0.25] (\i) at (-2*\i+4, 0) {};
            \fi
        }
        % draw lines
        \draw (-8,0) -- (-6,0);
        \draw (-6,0) -- (-4,0);
        \draw (-4,0) -- (-2,0);
        \draw (-2,0) -- (0,0);
        \draw[dashed, red] (-6,0) -- (-8,-2);
        \draw (-8,0) -- (-10,0);
        \draw (-10,0) -- (-12,0);
        \draw (-12,0) -- (-14,0);
        % add labels
        \node[scale=0.75] at (-9,-2) {$u_i$};
        \node[scale=0.75] at (-6,0) {$v_j$};
        % draw nodes
        \foreach \i in {1,...,6} {
            \pgfmathtruncatemacro{\result}{mod(\i+3,2)}
            \ifnum\result=1\relax
                \node[circle, fill=black, scale=0.25] (\i) at (2*\i-4, 0) {};
            \else
                \node[circle, fill=black, scale=0.25] (\i) at (2*\i-4, 0) {};
            \fi
        }
        % draw lines
        \draw (-8,0) -- (-6,0);
        \draw (-6,0) -- (-4,0);
        \draw (-4,0) -- (-2,0);
        \draw (-2,0) -- (0,0);
        \draw[dashed, red] (-6,0) -- (-8,-2);
        \draw (-8,0) -- (-10,0);
        \draw (-10,0) -- (-12,0);
        \draw (-12,0) -- (-14,0);
        % add labels
        \node[scale=0.75] at (-9,-2) {$x_{\ell}$};
        \node[scale=0.75] at (-6,0) {$w_k$};
    \end{tikzpicture}
\end{figure}
\end{document}