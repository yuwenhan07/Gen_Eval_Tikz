documentclass[tikz]{standalone}
\usetikzlibrary{matrix}

\begin{document}
    \begin{tikzpicture}[scale=.8]
        \def\s{20} % chunk size in pixels
        \def\d{1.2} % spacing between chunk nodes
        
        % Headers for in-use chunks
        \draw[dashed, line width=.25mm] (-0.4*\s-\d, -1.2*\d) rectangle ++(\s+\d,\s);
        \node at (\d/2, \s+\d/2) {header has P = 1};
        
        \node[matrix, minimum width=\s, minimum height=\d, row sep=-\pgflinewidth, column sep=-\pgflinewidth, nodes={minimum height=\d}] at (-\d/2, 0) {
            Chunk 0: In-use & \phantom{header has P = 1}\\
            Chunk 1: In-use & \phantom{header has P = 1}\\
            Chunk 2: Free & \phantom{header has P = 1}\\
            Chunk 3: In-use & \phantom{header has P = 0}\\
            Chunk 100: In-use & \phantom{header has P = 1}\\
            Chunk 101: Free & \phantom{header has P = 1}\\
            Chunk 103: In-use & \phantom{header has P = 0}\\
            Chunk 1000: In-use & \phantom{header has P = 1}\\
            Chunk 1001: Free & \phantom{header has P = 1}\\
            Chunk 1002: In-use & \phantom{header has P = 0}\\
            Chunk 1002: In-use & \phantom{header has P = 1}\\
        };
        % Headers for free chunks
        \node at (\d/2, 4*\d+\s+\d/2) {\dots in-use\dots};
        \node at (\d/2, 6*\d+\s+\d/2) {\dots in-use\dots};
        \node at (\d/2, 9*\d+\s+\d/2) {\dots in-use\dots};
    \end{tikzpicture}
\end{document}