\documentclass{article}
\usepackage{amsmath, amssymb}
\usepackage{tikz}
\usetikzlibrary{arrows.meta,bending}
\begin{document}

\begin{figure}[h]
    \centering
    \begin{tikzpicture}[scale=1.5,>=stealth]
        \draw[->](0,0)--(0,-3.2)node[left]{$x$};
        \draw[->](0,0)--(4.7,-1)node[right]{$y$};
        \draw[->](0,0)--(2.5,3)node[right]{$z$};

        \draw[red](-0.1,0)--++(180+atan(3/2),sqrt(13)/2);
        \draw[red](4.6,0)--++(-atan(1),1);

        \node[right] at (-0.1,0) {$L$};
        \node[left] at (4.6,0) {$L$};
        \node[right] at (2.5,3) {$L$};

        \node[above] at (1.25,0) {$L$};

        \draw[->][red](1.25,0)--++(atan(1/2),-sqrt(5)/2) node[left]{$i$};
        \draw[->][red](1.25,0)--++(-atan(1/2),sqrt(5)/2) node[right]{$j$};

        \draw[->][red](2.5,3)--++(-90+atan(3/2),sqrt(13)/2) node[above]{$k$};
    \end{tikzpicture}
    \qquad
    =
    \sum_l m^{(z,k)}_{(x,i),(y,j),l}
    \begin{tikzpicture}[scale=1.5,>=stealth]
        \draw[->](0,0)--(0,-3.2)node[left]{$x$};
        \draw[->](0,0)--(4.7,-1)node[right]{$y$};
        \draw[->](0,0)--(2.5,3)node[right]{$z$};

        \draw[red](-0.1,0)--++(180+atan(3/2),sqrt(13)/2);
        \draw[red](4.6,0)--++(-atan(1),1);

        \node[right] at (-0.1,0) {$L$};
        \node[left] at (4.6,0) {$L$};
        \node[right] at (2.5,3) {$L$};

        \node[right] at (1.25,0) {$l$};

        \draw[->][red](1.25,0)--++(atan(1/2),-sqrt(5)/2);
        \draw[->][red](1.25,0)--++(-atan(1/2),sqrt(5)/2);
    \end{tikzpicture}
\end{figure}

\end{document}