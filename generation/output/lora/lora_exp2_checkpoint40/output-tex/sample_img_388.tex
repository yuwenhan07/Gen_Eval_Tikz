\documentclass[a4paper]{article}
\usepackage[T1]{fontenc}
\usepackage[utf8]{inputenc}
\usepackage{amsmath,amsfonts,amssymb,bm,mathrsfs,xcolor}
\usepackage{xcolor}
\usepackage{pgfplots}
\pgfplotsset{compat=newest}
\usepgfplotslibrary{colorbrewer}
\usetikzlibrary{arrows.meta,babel,layers,positioning}

\begin{document}

\begin{figure}[h]
	\centering
	\begin{tikzpicture}
		\begin{axis}[
			scale = 0.7,
			height = 8cm,
			width = 8cm,
			xlabel = {$\beta_{M}$},
			ylabel = {$\rho_T$},
			legend style={draw=none, font=\footnotesize},
			yticklabel style={/pgf/number format/fixed,/pgf/number format/precision=1},
			xticklabel style={/pgf/number format/fixed,/pgf/number format/precision=1},
			legend pos = north east,
			minor tick num = 1,
			grid = both,
			xmin = 0,
			xmax = 5,
			ymin = 0,
			ymax = 5,
			xtick={0, 1, ..., 5},
			ytick={0, 1, ..., 5},
			clip=false,
			major tick length=2pt,
			ylabel near ticks,
			xlabel near ticks,
			major grid style={dotted},
			minor grid style={dotted}
		]
		
		\addplot+[no markers, red, thick] {exp(-1/x)};
		
		\addplot+[no markers, dashed, thick, red] coordinates {(0.69,0) (0.69,5)} node [pos=0.27, anchor=south west, rotate=90] {\textbf{Impossible recover}};
		\addplot+[no markers, dashed, thick, red] coordinates {(0.69,0) (5,0)} node [pos=0.27, anchor=north west, rotate=-90] {\textbf{Impossible recover}};
		
		\coordinate (origin) at (axis cs: 0, 0);
		
		\draw[dashed,thin,gray] (origin) -- ++(axis direction cs: 1,0) coordinate (right axisline cs: 0) ;
		\draw[dashed,thin,gray] (origin) -- ++(axis direction cs: 0,-1) coordinate (below axisline cs: 0) ;
		
		\begin{scope}[on background layer]
			\fill [red!30,domain=0.7:5,samples=100] (right axisline cs: 0) -- plot ({\x},{exp(-1/\x)}) |- (below axisline cs: 0) -- cycle;
		\end{scope}
		
		\addplot+[no markers, dashed, thick, red] coordinates {(0.69,0) (0.69,5)} node [pos=0.27, anchor=south west, rotate=90] {\textbf{Impossible recover}};
		\addplot+[no markers, dashed, thick, red] coordinates {(0.69,0) (5,0)} node [pos=0.27, anchor=north west, rotate=-90] {\textbf{Impossible recover}};
		\addlegendentry{$\zeta=0$ (phase transition)}
		\addlegendentry{$\beta_M=(\frac{c_1c_2}{1-c_3})^{1/4}$}
		
		
		\node[right] at(axis cs:5,5) {$\zeta$};
		\node[above right] at(axis cs:5,5) {$1.0$};
		\node[below right] at(axis cs:5,0) {$0.0$};
		
		\end{axis}
		\end{tikzpicture}
		\caption{\textbf{Asymptotic alignment} $\zeta^+=\max(\zeta,0)$ between the signal $\vy$ and the dominant eigenvector of $\rmT^{(2)}\rmT^{(2)\top}$, as defined in Theorem~\ref{thm:spike2}, with $c_1=\frac{1}{2}$, $c_2=\frac{1}{3}$ and $c_3=\frac{1}{6}$. The curve $\zeta=0$ is the position of the \textbf{phase transition} between the impossible detectability of the signal (below) and the presence of an isolated eigenvalue in the spectrum of $\rmT^{(2)}\rmT^{(2)\top}$ with corresponding eigenvector correlated with the signal (above). It has an asymptote $\beta_M=(\frac{c_1c_2}{1-c_3})^{1/4}$, represented by the red dashed line, as $\rho_T\rightarrow+\infty$.}
		\label{fig:T2T2TildeSPNphaseTransition}
	\end{figure}
\end{document}