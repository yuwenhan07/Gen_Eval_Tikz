\documentclass{amsart}
\usepackage{tikz}
\usetikzlibrary{backgrounds}
\begin{document}

\begin{figure}[htb]
\centering
\begin{tikzpicture}[scale=0.75]
    % draw the squares and the diagonals
    \foreach \i in {0,...,6}{
        \draw[fill=gray!10,draw=gray,thin] (\i*90+90/2,-4) rectangle ++(90,-4);
        \draw[draw=black,very thick,dotted] (\i*90+90/2, -4) -- (\i*90+90/2+90,-4);
        \draw[draw=black,very thick,dotted] (\i*90+90/2+90, -4) -- (\i*90+90/2+90+90,-4);
        \draw[draw=black,very thick,dotted] (\i*90+90/2+90+90, -4) -- (\i*90+90/2+90+90+90,-4);
    }
    \foreach \i in {0,...,6}{
        \draw[fill=gray!20,draw=gray,thin] (\i*90+90/2+90,-4) rectangle ++(90,-4);
        \draw[draw=black,very thick,dotted] (\i*90+90/2+90, -4) -- (\i*90+90/2+90+90,-4);
        \draw[draw=black,very thick,dotted] (\i*90+90/2+90+90, -4) -- (\i*90+90/2+90+90+90,-4);
        \draw[draw=black,very thick,dotted] (\i*90+90/2+90+90+90, -4) -- (\i*90+90/2+90+90+90+90,-4);
    }

    % draw the triangles and the edges
    \draw[fill=gray,draw=black,thin] (1.5,0) -- (6*90+1.5,0) -- (6*90+4.5,0) -- cycle;
    \draw[fill=gray,draw=black,thin] (-1.5,0) -- (-6*90+1.5,0) -- (-6*90+4.5,0) -- cycle;
    \foreach \i in {1,...,6}{
        \draw[draw=black,very thick,dotted] (\i*90+1.5,0) -- (\i*90+4.5,0);
        \draw[draw=black,very thick,dotted] (\i*90+4.5,0) -- (\i*90+7.5,0);
        \draw[draw=black,very thick,dotted] (\i*90+7.5,0) -- (\i*90+10.5,0);
    }
    \foreach \i in {0,...,5}{
        \draw[draw=black,very thick,dotted] (\i*90+10.5,0) -- (\i*90+13.5,0);
    }

    % draw the corners
    \foreach \i in {0,...,5}{
        \node at (\i*90+90/2+10.5,0) [circle,inner sep=1pt,fill] {};
        \node at (\i*90+90/2+13.5,0) [circle,inner sep=1pt,fill] {};
    }
    \node at (6*90+90/2+10.5,0) [circle,inner sep=1pt,fill] {};
    \node at (6*90+90/2+13.5,0) [circle,inner sep=1pt,fill] {};

    % draw the labels
    \foreach \i in {0,...,6}{
        \node at (\i*90+90/2+90+2.5,-4.5) {$\i+1$};
    }
    \node at (6*90+90/2+11.5,-4.5) {$8$};
    \node at (6*90+90/2+14.5,-4.5) {$9$};
    \node at (7*90+90/2+9,-4.5) {$m$};

    % draw the circles
    \foreach \i in {0,...,5}{
        \node[circle,inner sep=1pt,fill] at (\i*90+90/2+13.5,-4) {$\i$};
    }
    \node[circle,inner sep=1pt,fill] at (6*90+90/2+13.5,-4) {$\vdots$};

    % draw the label for k
    \node at (6*90+90/2+4.5,-1) {$k$};
    \node at (6*90+90/2+4.5,-1) [anchor=north east] {$\text{squares}$};
    \node at (6*90+90/2+4.5,-1) [anchor=south east] {$\text{}$};
    \node at (6*90+90/2+4.5,-1) [anchor=east] {$\text{}$};

    % draw the axis lines
    \draw[thick] (0,0) -- node[above=1em] {$7$} (0,-6);
    \draw[thick] (7*90+90/2+9,-6) -- node[below=1em] {$1$} (7*90+90/2+9,0);

    % draw the arrows
    \draw[->,>=latex,thick] (0,-6) -- (7*90+90/2+9,-6);

\end{tikzpicture}
\caption{The finned 2-complex obtained from the Cartesian product $K_{1,m}\Box P_{k+1}$ when $m$ is even. The boundary $C_{km}$ is capped with a polygonal region to form a 2-circlet.}
\label{fig:circlet}
\end{figure}
\end{document}