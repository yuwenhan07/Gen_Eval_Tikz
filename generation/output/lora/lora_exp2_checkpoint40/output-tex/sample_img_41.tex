documentclass[tikz,border=3mm]{standalone}
\usepackage{amsmath}

\begin{document}

\begin{tikzpicture}[bullet/.style={circle,draw=black,fill=#1,inner sep=2pt}]
\tikzset{bullet/.default={}}
\newcommand{\numpoints}{14}
\foreach \i in {0,...,\numexpr\numpoints-1} {
    \pgfmathtruncatemacro{\value}{mod(\i, 2)==0 ? 1 : -1}
    \node[bullet=\value] at ({\i/5},1){};
}
\foreach \i in {-1,0,1} {
    \node at ({\i*2/5},1.5) {\i};
}
\foreach \i in {0,...,\numexpr\numpoints-1} {
    \pgfmathtruncatemacro{\value}{mod(\i+1, 2)==0 ? 1 : -1}
    \node[bullet=\value] at ({\i/5},2){};
}
\draw ({-0.5/5},2)--({\numpoints/5},2);
\draw ({-0.5/5},1)--({\numpoints/5},1);
\node[anchor=north west] at ({-0.25/5},2){$\lambda_{i,k}(1)$};
\node[anchor=south west] at ({-0.25/5},1){$\lambda_{i,k}(1)$};
\node[anchor=north east] at ({\numpoints/5},{1+0.6}){$\lambda_{i,k}(1)$};
\node[anchor=south east] at ({\numpoints/5},{2+0.6}){$\lambda_{i,k}(1)$};
\end{tikzpicture}
\end{document}