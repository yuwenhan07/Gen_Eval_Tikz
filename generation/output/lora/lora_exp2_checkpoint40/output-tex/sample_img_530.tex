documentclass[tikz,border=2mm]{standalone}
\tikzset{
  declare function={
    f(\x)=\x/(\x+1); 
    g(\x,\y)=(\x)*exp(-pow((\y)/\x, 1.5));  
    xDelta=\Dt;
    alpha=-1/3;
    delta=-1/4;
    beta=-1;
    b0=6;    
    b1=-2;
    b2=-1;
    b3=1;
    b4=1;
    b5=-1;
    b6=1;
    b7=-1;
    b8=1;
    b9=1;
    b10=-1;
    b11=-1;
    b12=1;
    b13=-1;
    b14=1;
    b15=-1;
    b16=1;
    b17=-1;
    b18=1;
    b19=-1;
    b20=1;
    b21=-1;
    B1=(b0)/(delta*(delta+beta));
    B2=(b1)/(beta*(beta+delta));   
    }
}

\begin{document}
\begin{tikzpicture}[font=\small]
\draw[-latex] (-1,-4) -- (5.5,0) node[right]{$x$};
\draw[-latex] (0,-4) -- (0,13);
\path foreach \X in {-1,...,5}
{(0,{(\X>1 ? g(\X,xDelta)*exp(-pow(\X*xDelta/(delta*(delta+beta)), alpha)) : -g(\X,xDelta)*exp(-pow(\X*xDelta/(beta*(beta+delta)), alpha)))}) node (P-\X) {}};
\draw[thick,red,samples=100,domain=0.2:4] plot ({\X},{f(P-\X)});
\foreach \X in {1,...,5}{
\draw (P-\X) node {$\bullet$};}
\foreach \X/\Y in {1/-2.5,2/4.5,3/3.5}{
\draw (P-\X) node[circle,fill,inner sep=2pt,label=above:$\psi(x)$] {};
\draw (P-\X) node [left]{$\Y$};}
\draw (P-1) -- (P-2) -- (P-3);
\draw[thick,blue!50!gray,samples=100,domain=3:4] plot ({\X},{exp(-pow(\X*(xDelta+(delta+beta)*exp(-1/alpha))/delta, alpha))}); 
\draw (P-3) -- ++ (0.2,0) node[right]{$\xi_{\Dt}$};
\draw (P-3) -- ++ (0.3,0) node[right]{$\xi_{\Dt+\epsilon}$};  
\end{tikzpicture}
\end{document}