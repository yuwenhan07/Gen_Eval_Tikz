\documentclass[border=0pt,tikz]{standalone}
\usepackage{amsmath,mathtools}
\usepackage{xcolor}
\usepackage{tikz}
\usetikzlibrary{decorations.pathreplacing,backgrounds}
\tikzset{myarrow/.style={->,>=latex}}
\newcommand{\N}{\mathbb{N}}
\begin{document}

\begin{tikzpicture}[scale=0.9]
    \foreach \x in {0,...,8}{
        \draw[line width=0.2mm,black!60!white] (\x,-0.5)--(\x+0.5,-0.5);
    }
    \foreach \x in {0,1,...,8}{
        \draw[fill=black!5] (\x,-0.5) node[above] {\x} circle[radius=0.04];
    }

    \draw[-latex,line width=0.8mm] (0,-1)--(0,8)node[anchor=south west]{$\varphi_t(x)$};
    \draw[-latex,line width=0.8mm] (-0.5,0)--(9,-0.5)node[anchor=north east]{$x$};
    \draw[line width=0.8mm,red] (5,0)--(5,7)node[anchor=south west]{$\varphi_t(5)$};

    \foreach \x in {0,...,8}{
        \pgfmathtruncatemacro\result{\x*(mod(\x,2)==1)?(-1)**floor((\x-1)/2):-1};
        \draw[gray!50,fill=gray!50] ({\x},{(\x*(mod(\x,2)==1)?(-1)**floor((\x-1)/2):-1)*sin(\x*pi/2)} )--++(0.5,0) -- ++(0,{(\x*(mod(\x,2)==1)?(-1)**floor((\x-1)/2):-1)*sin(\x*pi/2)}) -- cycle;
    }
\end{tikzpicture}

\end{document}