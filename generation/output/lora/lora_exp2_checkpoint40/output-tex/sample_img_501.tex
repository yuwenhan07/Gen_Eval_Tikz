documentclass[tikz,border=3mm]{standalone}
\usetikzlibrary{matrix}

\tikzset{mymat/.style={
    matrix,
    draw,
    row sep=0pt,
    column sep=0pt,
    inner sep=0pt,
    outer sep=0pt,
    nodes={minimum size=6pt,text depth=0.25ex,text height=0.5ex}
},
every label/.append style={text depth=0.25ex,text height=0.5ex}
}

\begin{document}
\begin{tikzpicture}[node distance=1em]
    \matrix[mymat,label=G/$H_{1}$] (mat1) {%
        |(e)|1\\
    };
    \matrix[mymat,label=G/$H_{2}$,below=of mat1.south east] (mat2) {
        |(e)|1&2&3&4&5&6&7&8\\
        |(e)|9&10&11&12&13&14&15&16\\
    };
    \matrix[mymat,label=G/$H_{3}$,above=of mat2.north east] (mat3) {
        |(e)|1&2\\
    };

    \node[red,text width=10em,anchor=east,xshift=-5em,yshift=-3em] at (mat2.south west) {0};
    \node[red,text width=10em,anchor=east,xshift=-5em,yshift=-3em] at (mat2.north west) {0};
    \node[red,text width=10em,anchor=east,xshift=-5em,yshift=-3em] at (mat3.south west) {0};
    \node[red,text width=10em,anchor=east,xshift=-5em,yshift=-3em] at (mat3.north west) {0};
    \node[red,text width=10em,anchor=east,xshift=-5em,yshift=-3em] at (mat1.south west) {0};
    \node[red,text width=10em,anchor=east,xshift=-5em,yshift=-3em] at (mat1.north west) {0};

    \draw[red,very thick] (-6,-2) -- (8,-2);
    \foreach \x in {-2,...,10}
        \draw (-6,\x) node[above] {\x} -- ++(0,.1pt);
    \foreach \x in {-2,...,10}
        \draw (-6+\x,-2) node[right] {\x} -- ++(.1pt,0);
\end{tikzpicture}
\end{document}