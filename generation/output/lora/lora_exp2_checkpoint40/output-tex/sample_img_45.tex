document
\documentclass{article}
\usepackage{tikz}
\usetikzlibrary{backgrounds}
\begin{document}
\tikzset{
    mynode/.style={circle,inner sep=0pt,minimum size=4mm,fill=#1},
    myline/.style={thick,draw=#1},
    mypoint/.style={circle,inner sep=0pt,minimum size=1mm,fill=black},
}
\begin{tikzpicture}[scale=3,transform shape]
    \begin{scope}[on background layer]
        \filldraw[fill=gray!20,rounded corners] (-.2,-.2) rectangle (.9,.2);
        \draw[myline=gray!40] (0,-.2) rectangle (1,.2);
        \draw[myline=gray!40] (0,0)--(1,0);
        \foreach \i in {0,1,2}{
            \draw[myline=gray!40] (\i/3,{mod(\i,3)*.2})--(\i/3+.3,{mod(\i,3)*.2});
            \draw[myline=gray!40] ({mod(\i,3)*.2},\i/3)--({mod(\i,3)*.2},\i/3+.3);
        }
    \end{scope}
    \draw[myline=red] (0,0)--(1,0);
    \foreach \i in {0,...,6}{
        \pgfmathtruncatemacro{\j}{\i+1}
        \pgfmathtruncatemacro{\k}{mod(\i+1,7)}
        \ifnum\i>0
            \draw[myline=black] ({mod(\i,7)/3},{mod(\i,3)*.2})--({mod(\k,7)/3},{mod(\k,3)*.2});
        \fi
        \ifnum\i<5
            \draw[myline=black] ({mod(\i,3)*.2},{mod(\i,7)/3})--({mod(\j,3)*.2},{mod(\j,7)/3});
        \fi
        \draw[mypoint,fill=white] ({mod(\i,3)/3},{mod(\i,3)*.2}) circle[];
        \draw[mypoint,fill=white] ({mod(\i,3)*.2},{mod(\i,3)/3}) circle[];
        \node[mynode=red] at ({mod(\i,3)/3},{mod(\i,3)*.2}) {$A$};
        \node[mynode=red] at ({mod(\i,3)*.2},{mod(\i,3)/3}) {$B$};
    }
    \draw[myline=black,shorten >=-.8mm] (-.1,0)--(-.1,.2);
    \draw[myline=black,shorten >=-.8mm] (-.1,.2)--(.1,.2);
    \draw[myline=black,shorten >=-.8mm] (.1,.2)--(.1,-.2);
    \draw[myline=black,shorten >=-.8mm] (.1,-.2)--(-.1,-.2);
\end{tikzpicture}
\begin{tikzpicture}[scale=3,transform shape]
    \begin{scope}[on background layer]
        \filldraw[fill=gray!20,rounded corners] (-.2,-.2) rectangle (.9,.2);
        \draw[myline=gray!40] (0,-.2) rectangle (1,.2);
        \draw[myline=gray!40] (0,0)--(1,0);
        \foreach \i in {0,1,2}{
            \draw[myline=gray!40] (\i/3,{mod(\i,3)*.2})--(\i/3+.3,{mod(\i,3)*.2});
            \draw[myline=gray!40] ({mod(\i,3)*.2},\i/3)--({mod(\i,3)*.2},\i/3+.3);
        }
    \end{scope}
    \draw[myline=red] (0,0)--(1,0);
    \draw[myline=red,domain=0:.7*pi,samples=100] plot (\x:{tan(\x)});
    \foreach \i in {0,...,6}{
        \pgfmathtruncatemacro{\j}{\i+1}
        \pgfmathtruncatemacro{\k}{mod(\i+1,7)}
        \ifnum\i>0
            \draw[myline=black] ({mod(\i,7)/3},{mod(\i,3)*.2})--({mod(\k,7)/3},{mod(\k,3)*.2});
        \fi
        \ifnum\i<5
            \draw[myline=black] ({mod(\i,3)*.2},{mod(\i,7)/3})--({mod(\j,3)*.2},{mod(\j,7)/3});
        \fi
        \draw[mypoint,fill=white] ({mod(\i,3)/3},{mod(\i,3)*.2}) circle[];
        \draw[mypoint,fill=white] ({mod(\i,3)*.2},{mod(\i,3)/3}) circle[];
        \node[mynode=red] at ({mod(\i,3)/3},{mod(\i,3)*.2}) {$A$};
        \node[mynode=red] at ({mod(\i,3)*.2},{mod(\i,3)/3}) {$B$};
    }
    \draw[myline=black,shorten >=-.8mm] (-.1,0)--(-.1,.2);
    \draw[myline=black,shorten >=-.8mm] (-.1,.2)--(.1,.2);
    \draw[myline=black,shorten >=-.8mm] (.1,.2)--(.1,-.2);
    \draw[myline=black,shorten >=-.8mm] (.1,-.2)--(-.1,-.2);
\end{tikzpicture}
\end{document}