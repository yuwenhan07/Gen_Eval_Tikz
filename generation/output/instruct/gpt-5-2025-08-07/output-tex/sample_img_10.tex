\documentclass[border=2pt]{standalone}
\usepackage{amsmath}
\usepackage{xcolor}
\usepackage{tikz}
\usepackage{quantikz} % Quantum circuits on TikZ

% Optional: Uniform line width
\tikzset{every picture/.style={line width=0.8pt}}

\begin{document}
\begin{quantikz}[row sep=1.2em, column sep=1.6em]
% 1st row (top wire)
\lstick{} &
  \gate{$R_y(\theta_1)$}
  % Group box starts here, spans 4 wires and 5 columns to the right
  \gategroup[wires=4,steps=5,
    style={dashed, rounded corners, inner xsep=6pt, inner ysep=4pt},
    label style={label position=right, anchor=west},
    label={$\times D$}]{}
  & \ctrl{1} & \qw      & \qw      & \targ{} & \qw \\
% 2nd row
\lstick{} &
  \gate{$R_y(\theta_2)$}
  & \targ{} & \ctrl{1} & \qw      & \qw     & \qw \\
% 3rd row
\lstick{} &
  \gate{$R_y(\theta_3)$}
  & \qw     & \targ{}  & \ctrl{1} & \ctrl{-2} & \qw \\
% 4th row (bottom wire)
\lstick{} &
  \gate{$R_y(\theta_4)$}
  & \qw     & \qw
  % The empty control point (\octrl) is used at this location as seen in the reference image
  & \octrl{-1}
  & \ctrl{-3} & \qw
\end{quantikz}
\end{document}