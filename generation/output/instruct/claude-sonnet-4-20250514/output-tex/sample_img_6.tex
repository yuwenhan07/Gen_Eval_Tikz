\documentclass[tikz,border=3mm]{standalone}
\usepackage{tikz}
\usepackage{amsmath}
\usetikzlibrary{shapes.geometric, arrows.meta, positioning, fit}

\begin{document}
\begin{tikzpicture}[
    scale=1,
    node distance=0.6cm and 1.8cm,
    every node/.style={font=\small},
    filled/.style={circle, fill=black, inner sep=1.5pt, minimum size=3pt},
    hollow/.style={circle, fill=white, draw=black, inner sep=1.5pt, minimum size=3pt},
    input/.style={font=\small},
    output/.style={font=\small},
    ellipse/.style={ellipse, draw=black, minimum width=1.2cm, minimum height=6cm},
    connection/.style={draw=black, line width=0.4pt}
]

% Input nodes (left side)
\foreach \i in {1,...,9} {
    \node[input] (x\i) at (0, {4.8-0.6*(\i-1)}) {$x_{\i}$ $\circ$};
}

% Ellipse regions and internal nodes
\foreach \col in {1,...,6} {
    % Create nodes within each ellipse
    \foreach \row in {1,...,9} {
        \pgfmathsetmacro\ypos{4.8-0.6*(\row-1)}
        \pgfmathsetmacro\xpos{1.8*\col}
        
        % Determine node type based on position (filled or hollow)
        \ifnum\col=1
            \node[filled] (n\col\row) at (\xpos, \ypos) {};
        \else
            \pgfmathtruncatemacro\randint{rnd*2} % Randomly choose between filled or hollow
            \ifnum\randint=0
                \node[filled] (n\col\row) at (\xpos, \ypos) {};
            \else
                \node[hollow] (n\col\row) at (\xpos, \ypos) {};
            \fi
        \fi
    }
    
    % Draw ellipse boundary
    \node[ellipse, fit={(n\col1) (n\col9)}, inner sep=0pt] (ellipse\col) {};
}

% Connect input nodes to internal nodes
\foreach \i in {1,...,9} {
    \foreach \col in {1,...,6} {
        \draw[connection] (x\i) .. controls +(right:0.5cm) and +(left:0.5cm) .. (n\col\i);
    }
}

% Output labels at the bottom
\foreach \i in {1,...,6} {
    \node[output] at ({1.8*\i}, -0.8) {$\i K_9$};
    \node[output] at ({1.8*\i}, -1.6) {$\i$};
}

% Horizontal lines for clarity
\draw[gray!50] ([yshift=1cm]current bounding box.north west) -- ([yshift=1cm]current bounding box.north east);
\draw[gray!50] ([yshift=-1cm]current bounding box.south west) -- ([yshift=-1cm]current bounding box.south east);

\end{tikzpicture}
\end{document}