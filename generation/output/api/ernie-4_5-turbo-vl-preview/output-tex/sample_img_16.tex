\documentclass[preview,multi]{standalone}
\usepackage{tikz}
\usepackage{amsmath}

\begin{document}
\begin{figure}
    \begin{tikzpicture}[scale=0.6]
        % Grid and labels for C1
        \draw (0,0) grid (6,4);
        \node at (-1,2) {$\mathcal{C}_1$};
        \foreach \i in {1,...,4} {
            \node at (-0.5,4.5-\i) {$\succ_{\i}$};
        }
        \foreach \i in {1,...,4} {
            \foreach \j in {1,...,6} {
                \node at (\j-0.5,4.5-\i) {\ifnum\j<5 \ifnum\j<\i 1\else\ifnum\j=\i 2\else\ifnum\j=\i+1 3\else\ifnum\j=\i+2 4\else \the\numexpr\j+6\relax\fi\fi\fi\fi}; % Simplified content for illustration
            }
        }
        % Top-ranked alternative for C1
        \node at (0.5,3.5) {1};
        \node at (0.5,2.5) {2};
        \node at (0.5,1.5) {3};
        \node at (0.5,0.5) {4};
        % Alternatives 11,12 for C1
        \node at (4.5,3.5) {11};
        \node at (5.5,3.5) {12};
        \node at (4.5,2.5) {11};
        \node at (5.5,2.5) {12};
        \node at (4.5,1.5) {11};
        \node at (5.5,1.5) {12};
        \node at (4.5,0.5) {11};
        \node at (5.5,0.5) {12};
        
        % Grid and labels for C2
        \draw (0,-6) grid (6,-2);
        \node at (-1,-4) {$\mathcal{C}_2$};
        \foreach \i in {5,...,8} {
            \node at (-0.5,-6.5+(\i-5)) {$\succ_{\i}$};
        }
        % Content for C2 similar to C1, omitted for brevity
        
        % Grid and labels for C3
        \draw (0,-12) grid (6,-8);
        \node at (-1,-10) {$\mathcal{C}_3$};
        \foreach \i in {9,...,12} {
            \node at (-0.5,-12.5+(\i-9)) {$\succ_{\i}$};
        }
        % Top-ranked alternative for C3
        \node at (0.5,-8.5) {9};
        % ... (additional content for C3, including symmetry and specific alternatives)
        
        % Grid and labels for C4
        \draw (0,-18) grid (6,-14);
        \node at (-1,-16) {$\mathcal{C}_4$};
        \foreach \i in {13,...,16} {
            \node at (-0.5,-18.5+(\i-13)) {$\succ_{\i}$};
        }
        % Top-ranked alternative for C4
        \node at (0.5,-14.5) {13};
        % ... (additional content for C4)
        
        % Vertical labels for t and t^2
        \node at (3,-19) {$t$};
        \node at (6,-19) {$t^2$};
        
        % Right side labels for C3 parts
        \draw (7,-8) grid (8,-6);
        \node at (7.5,-6.5) {17};
        \node at (8.5,-6.5) {18};
        \node[rotate=90] at (9,-7) {Upper half of $\mathcal{C}_3$};
        
        \draw (7,-12) grid (8,-10);
        \node at (7.5,-10.5) {17};
        \node at (8.5,-10.5) {18};
        \node[rotate=90] at (9,-11) {Lower half of $\mathcal{C}_3$};
    \end{tikzpicture}
    \caption{An example of our lower bound construction in Theorem~1 where $m = 18$. Hence, we set $t=4$ and $n=t^2=16$ resulting in the profile $P$ shown above. In every agent's ranking (horizontal bars), the example mentions the top-ranked alternative. Additionally, we marked alternatives $11$ and $12$ in every ranking in order to showcase the symmetry of $P$. Alternatives $17$ and $18$ are shown for two agents of every cohort. Notice that these alternatives appear in the exact same positions of every agent's ranking.}
\end{figure}
\end{document}