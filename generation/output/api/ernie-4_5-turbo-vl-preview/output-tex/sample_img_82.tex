\documentclass{standalone}
\usepackage{tikz}
\begin{document}
\begin{tikzpicture}
    % Define epsilon value
    \def\e{1}
    
    % Define points
    \coordinate (0) at (0, 0);
    \coordinate (1) at (2*\e, 0);
    \coordinate (2) at (\e, {\sqrt(3)*\e});
    \coordinate (3) at (\e, {-\sqrt(3)*\e});
    \coordinate (G) at (\e, 0);
    \coordinate (H) at (\e, 0); % It's the same as point G in this context
    
    % Draw lines
    \draw (0) -- (1) -- (2) -- cycle;
    \draw (0) -- (1) -- (3) -- cycle;
    \draw[dashed] (2) -- (0) -- (3) -- (2) -- (1); % Dashed lines for the inner triangle connections
    \draw[dashed] (G) -- (2); % Not strictly necessary as G is on 2-0-3-2 path, but for clarity
    \draw[dashed] (G) -- (3);
    \draw[dashed] (G) -- (0);
    \draw[dashed] (G) -- (1);
    
    % Draw vertical line through G (same as H in this context)
    \draw[dashed] (2) -- (3);
    \draw (G) -- (H) node [midway, right] {\tiny $H$}; % Since G and H are the same here, it's just for label
    
    % Label points
    \node[above right] at (2) {$2$};
    \node[below right] at (3) {$3$}; % Added label for point 3
    \node[below left] at (0) {$0$};
    \node[below right] at (1) {$1$};
    \node[right] at (G) {$G$};
    
    % Label distances (optional)
    \node[above left] at (\e/2, {\sqrt(3)*\e/2}) {$C_1$};
    \node[above right] at ({\e + \e/2}, {\sqrt(3)*\e/2}) {$C_2$};
    \node[above] at (\e, {\sqrt(3)*\e}) {$C_3$}; % Adjusted to appear above point 2
\end{tikzpicture}
\end{document}