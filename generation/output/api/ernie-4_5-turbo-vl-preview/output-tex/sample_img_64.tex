\documentclass{standalone}
\usepackage{tikz}
\usetikzlibrary{arrows.meta}

\begin{document}

\begin{tikzpicture}[
    node distance=2cm and 3cm,
    >=Stealth,
    auto,
    compartment/.style={circle, draw, fill=yellow!20, minimum size=1cm}
]

% Nodes
\node[compartment] (S) at (0,0) {$S$};
\node[compartment, right=of S, yshift=2cm] (V1) {$V_1$};
\node[compartment, right=of V1] (V2) {$V_2$};
\node[compartment, right=of V2] (V3) {$V_3$};
\node[compartment, right=of S, yshift=-2cm] (E) {$E$};
\node[compartment, right=of E] (U) {$U$};
\node[compartment, left=of S, yshift=-2cm] (R) {$R$};

% Edges
\draw[->] (S) -- node[above] {$\beta$} (E);
\draw[->] (E) -- node[above, sloped] {$\tau + \lambda$} (U);
\draw[->] (U) -- node[above, sloped] {$\delta$} (R);
\draw[->] (R) -- node[above, sloped] {$\gamma$} (S);

\draw[->] (V1) -- node[above] {$\alpha_2$} (V2);
\draw[->] (V2) -- node[above] {$\alpha_3$} (V3);

\draw[->] (V1) -- node[above, sloped] {$\phi_2$} (S);
\draw[->] (V2) -- node[above, sloped] {$\phi_3$} (S);

\draw[->] (S) -- node[left] {$\alpha_1$} (V1);
\draw[->] (V1) -- node[left] {$\phi_1$} (S); % Correcting the direction and label position

% Wait annotations
\node[above=of V1, yshift=-1cm] {(Wait 28 days)};
\node[above=of V2, yshift=-1cm] {(Wait 180 days)};

\end{tikzpicture}

\end{document}