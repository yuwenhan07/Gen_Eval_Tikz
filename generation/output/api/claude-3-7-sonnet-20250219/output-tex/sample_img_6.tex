\documentclass{article}
\usepackage{tikz}
\usepackage{amsmath}

\begin{document}

\begin{tikzpicture}[scale=1.2]
    % Define coordinates for vertices
    \foreach \i in {1,...,6} {
        \foreach \j in {1,...,9} {
            \coordinate (v\i\j) at (\i, 10-\j);
        }
    }
    
    % Draw the axes
    \draw[thick] (-0.5,0) -- (-0.5,9.5);
    \draw[thick] (-0.5,0) -- (0.5,0);
    
    % Draw labels for rows (x values)
    \foreach \j in {1,...,9} {
        \node[left] at (-0.5, 10-\j) {$x_{\j-1}$};
        \node[circle, draw, inner sep=1.5pt, fill=white] at (-0.5, 10-\j) {};
    }
    
    % Draw the column labels (K_9 values)
    \foreach \i in {1,...,6} {
        \node[below] at (\i, -0.3) {$\i K_9$};
        \node[circle, draw, inner sep=1.5pt, fill=white] at (\i, -0.8) {};
        \node[below] at (\i, -1.3) {$\i$};
    }
    
    % Draw the ellipses around each column
    \foreach \i in {1,...,6} {
        \draw[thick] (\i,0.5) ellipse (0.3 and 9);
    }
    
    % Define the pattern for filled/unfilled vertices
    % This creates the Sierpiński pattern
    \foreach \i in {1,...,6} {
        \foreach \j in {1,...,9} {
            % Determine if the vertex should be filled or unfilled
            % Based on the binary representation of i-1 and j-1
            \pgfmathtruncatemacro{\iminusone}{\i-1}
            \pgfmathtruncatemacro{\jminusone}{\j-1}
            \pgfmathtruncatemacro{\bitand}{bitand(\iminusone,\jminusone)}
            
            \ifthenelse{\bitand=0}{
                % Filled vertex
                \node[circle, draw, inner sep=1.5pt, fill=black] at (v\i\j) {};
            }{
                % Unfilled vertex
                \node[circle, draw, inner sep=1.5pt, fill=white] at (v\i\j) {};
            }
        }
    }
    
    % Draw the connecting lines
    % These are the edges that connect vertices in different columns
    % Based on the Sierpiński pattern
    \foreach \i in {1,...,5} {
        \pgfmathtruncatemacro{\nexti}{\i+1}
        \foreach \j in {1,...,9} {
            \pgfmathtruncatemacro{\iminusone}{\i-1}
            \pgfmathtruncatemacro{\jminusone}{\j-1}
            \pgfmathtruncatemacro{\bitand}{bitand(\iminusone,\jminusone)}
            
            \pgfmathtruncatemacro{\nextminusone}{\nexti-1}
            \pgfmathtruncatemacro{\nextbitand}{bitand(\nextminusone,\jminusone)}
            
            \ifthenelse{\bitand=0 \AND \nextbitand=0}{
                \draw (v\i\j) -- (v\nexti\j);
            }{}
        }
    }
    
\end{tikzpicture}

\end{document}