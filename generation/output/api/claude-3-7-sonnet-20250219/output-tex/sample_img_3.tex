\documentclass{article}
\usepackage{tikz}
\usetikzlibrary{positioning}

\begin{document}

\begin{figure}
\centering
\begin{tikzpicture}[
    level distance=1cm,
    level 1/.style={sibling distance=2cm},
    level 2/.style={sibling distance=1.8cm},
    level 3/.style={sibling distance=1.6cm},
    level 4/.style={sibling distance=1.4cm},
    level 5/.style={sibling distance=1.2cm},
    level 6/.style={sibling distance=1cm},
    every node/.style={font=\normalsize}
]
% Root node
\node (q0) {$q_0$}
    child {node[left] {$c$}}
    child {node[right] {$a$}
        child {node[left] {$c$}}
        child {node[right] {$a$} 
            child {node[left] {$d$}}
            child {node[right] {$a$} 
                child {node[left] {$c/d$}}
                child {node[right] {$a$} 
                    child {node[left] {$c$}}
                    child {node[right] {$a$} 
                        child {node[left] {$c$}}
                        child {node[right] {$c$}}
                    }
                }
            }
        }
    };

% Add the state labels on the right side with dotted lines
\node (q1) at ($(q0)+(0.5,-1)$) {$q_1$};
\draw[dotted] (q1) -- ($(q0)!0.5!(q0-2)$);

\node (qk) at ($(q1)+(1.2,-1)$) {$q_k = q$};
\draw[dotted] (qk) -- ($(q0-2)!0.5!(q0-2-2)$);

\node (qkp) at ($(qk)+(0.8,-1)$) {$q_{k+p} = q$};
\draw[dotted] (qkp) -- ($(q0-2-2)!0.5!(q0-2-2-2)$);

\node (qk2p) at ($(qkp)+(0.8,-1)$) {$q_{k+2p} = q$};
\draw[dotted] (qk2p) -- ($(q0-2-2-2)!0.5!(q0-2-2-2-2)$);

\node (qk3p) at ($(qk2p)+(0.6,-1)$) {$q_{k+3p}$};
\draw[dotted] (qk3p) -- ($(q0-2-2-2-2)!0.5!(q0-2-2-2-2-2)$);

\end{tikzpicture}
\caption{Illustration for the proof of Proposition~\ref{twoexamples}. All inner nodes of the tree are labeled with~$a$. All leaves not shown in the illustration are labeled with $c$.}
\end{figure}

\end{document}