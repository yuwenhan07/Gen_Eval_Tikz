% Compile this as a standalone document
\documentclass{article}
\usepackage[margin=1in]{geometry}
\usepackage{quantikz}

\begin{document}

\begin{figure}[h!]
\centering
\begin{quantikz}[row sep={0.9cm,between origins}, column sep=0.9cm]
& \gate{$R_y(\theta_1)$}
  \gategroup[wires=4, steps=5, style={dashed, rounded corners, inner xsep=6pt, inner ysep=6pt}]{} 
  & \ctrl{1} & \qw       & \qw       & \targ{}   & \qw \\
& \gate{$R_y(\theta_2)$} 
  & \targ{}   & \ctrl{1}  & \qw       & \qw       & \qw \\
& \gate{$R_y(\theta_3)$} 
  & \qw       & \targ{}   & \ctrl{1}  & \qw       & \qw \\
& \gate{$R_y(\theta_4)$} 
  & \qw       & \qw       & \targ{}   & \ctrl{-3} & \rstick{$\times D$}
\end{quantikz}
\caption{An example of BasicEntangler ansatz. $R_y(\theta_i)$ is a rotation gate around the $y$-axis by angle $\theta_i$ applied to the $i$-th qubit ($i=1,2,3,4$). The circuit layer in the dashed box can be repeated $D$ times to increase the representational capacity of the ansatz.}
\end{figure}

\end{document}