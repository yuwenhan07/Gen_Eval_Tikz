Based on the provided image and description, here is the LaTeX/TikZ code that reconstructs the visual appearance of the lower bound construction example:
```latex
\documentclass[tikz,border=10pt]{standalone}
\usetikzlibrary{calc}
\begin{document}
\begin{tikzpicture}[>=latex]

% set length of each cell
\newlength\celllength
\setlength\celllength{8mm}

% set number of agents and alternatives
\newcount\agents
\agents=16
\newcount\alternatives
\alternatives=18

% margin
\newdimen\margin
\margin=0mm

% coordinates of the grid
\newdimen\GridX
\newdimen\GridY
\GridX=\the\numexpr\alternatives/2\relax\celllength
\GridY=\agents\celllength

% overall coordinates
\coordinate (O) at (0,0);
\coordinate (L) at (\GridX+\margin,0);
\coordinate (B) at (0,-\agents*\celllength-\margin);
\coordinate (R) at (\GridX+\margin,-\GridY-\margin);

% draw the grid
\draw (O) rectangle (R);

% labels
\node[anchor=east] at ($(O)+(0,\agents*\celllength/2)$){$C_1$\{$\ne1$}};
\node[anchor=east] at ($(O)+(0,\agents*\celllength/2-(\agents+1)*\celllength)$){$C_2$\{$\ne5$}};
\node[anchor=east] at ($(O)+(0,\agents*\celllength/2-2*(\agents+1)*\celllength)$){$C_3$\{$\ne9$}};
\node[anchor=east] at ($(O)+(0,\agents*\celllength/2-3*(\agents+1)*\celllength)$){$C_4$\{$\ne13$}};
\node at ($(O)+(\GridX/2,-(\agents+1)*\celllength/2)$){$t$};
\node at ($(O)+(\GridX/2,-3*(\agents+1)*\celllength/2)$){$t^2$};

% profiles

% P1
\foreach \i in {1,...,4} {
  \node at ($(O)+(0,-(\i-1)*\celllength-0.5*\celllength-\margin)$){$\i$};
  \draw ($(O)+(0,-(\i-1)*\celllength-0.5*\celllength-\margin)$)--++(\celllength,0);
}
\draw ($(O)+(0.5*\celllength,-\celllength)$)--++(0,-12*\celllength);
\draw ($(O)+(0.5*\celllength,-\celllength-11*\celllength)$)--++(0,\celllength);
\draw ($(O)+(0.5*\celllength,-\celllength-11*\celllength-1*\celllength)$)--++(\celllength,0);
\draw ($(O)+(0.5*\celllength,-\celllength-12*\celllength)$)--++(\celllength,0);
\node at ($(O)+(1.5*\celllength,-\celllength-11.5*\celllength)$){$11$};
\node at ($(O)+(2*\celllength,-\celllength-11.5*\celllength)$){$12$};

% P2
\foreach \i in {5,...,8} {
  \node at ($(O)+(0,-(\i-1-4)*\celllength-0.5*\celllength-\margin)$){$\i$};
  \draw ($(O)+(0,-(\i-1-4)*\celllength-0.5*\celllength-\margin)$)--++(\celllength,0);
}
\draw ($(O)+(3.5*\celllength,-\celllength-4*\celllength)$)--++(0,-4*\celllength);
\draw ($(O)+(3.5*\celllength,-\celllength-8*\celllength)$)--++(0,\celllength);
\draw ($(O)+(3.5*\celllength,-\celllength-8*\celllength-1*\celllength)$)--++(\celllength,0);
\draw ($(O)+(3.5*\celllength,-\celllength-9*\celllength)$)--++(\celllength,0);
\node at ($(O)+(4.5*\celllength,-\celllength-8.5*\celllength)$){$11$};
\node at ($(O)+(5*\celllength,-\celllength-8.5*\celllength)$){$12$};

% P3
\foreach \i in {9,...,12} {
  \node at ($(O)+(0,-(\i-1-8)*\celllength-0.5*\celllength-\margin)$){$\i$};
  \draw ($(O)+(0,-(\i-1-8)*\celllength-0.5*\celllength-\margin)$)--++(\celllength,0);
}
\draw ($(O)+(7*\celllength,-\celllength-8*\celllength)$)--++(0,-4*\celllength);
\draw ($(O)+(7*\celllength,-\celllength-12*\celllength)$)--++(0,\celllength);
\node at ($(O)+(7.5*\celllength,-\celllength-11.5*\celllength)$){$11$};
\node at ($(O)+(8*\celllength,-\celllength-11.5*\celllength)$){$12$};

% P4
\foreach \i in {13,...,16} {
  \node at ($(O)+(0,-(\i-1-12)*\celllength-0.5*\celllength-\margin)$){$\i$};
  \draw ($(O)+(0,-(\i-1-12)*\celllength-0.5*\celllength-\margin)$)--++(\celllength,0);
}
\draw ($(O)+(10.5*\celllength,-\celllength-12*\celllength)$)--++(0,-4*\celllength);
\draw ($(O)+(10.5*\celllength,-\celllength-16*\celllength)$)--++(0,\celllength);
\draw ($(O)+(10.5*\celllength,-\celllength-16*\celllength-1*\celllength)$)--++(\celllength,0);
\draw ($(O)+(10.5*\celllength,-\celllength-17*\celllength)$)--++(\celllength,0);
\node at ($(O)+(11.5*\celllength,-\celllength-16.5*\celllength)$){$11$};
\node at ($(O)+(12*\celllength,-\celllength-16.5*\celllength)$){$12$};

% alternatives 17 and 18
\foreach \i in {1,...,4} {
  \node at ($(L)+(0,-(\i-1)*\celllength-0.5*\celllength-\margin)$){$17$};
  \node at ($(L)+(0,-(\i-1)*\celllength-0.5*\celllength-\margin-0.5*\celllength)$){$18$};
}
\foreach \i in {5,...,8} {
  \node at ($(L)+(3.5*\celllength,-\celllength-4*\celllength-(\i-5)*\celllength-0.5*\celllength-\margin)$){$17$};
  \node at ($(L)+(3.5*\celllength,-\celllength-4*\celllength-(\i-5)*\celllength-0.5*\celllength-\margin-0.5*\celllength)$){$18$};
}
\foreach \i in {9,...,12} {
  \node at ($(L)+(7*\celllength,-\celllength-8*\celllength-(\i-9)*\celllength-0.5*\celllength-\margin)$){$17$};
  \node at ($(L)+(7*\celllength,-\celllength-8*\celllength-(\i-9)*\celllength-0.5*\celllength-\margin-0.5*\celllength)$){$18$};
}
\foreach \i in {13,...,16} {
  \node at ($(L)+(10.5*\celllength,-\celllength-12*\celllength-(\i-13)*\celllength-0.5*\celllength-\margin)$){$17$};
  \node at ($(L)+(10.5*\celllength,-\celllength-12*\celllength-(\i-13)*\celllength-0.5*\celllength-\margin-0.5*\celllength)$){$18$};
}

% Upper half of C3
\node at ($(R)+(0