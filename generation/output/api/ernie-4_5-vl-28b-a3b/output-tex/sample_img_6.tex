\documentclass[tikz,border=10pt]{standalone}
\usetikzlibrary{calc}
\begin{document}
\begin{tikzpicture}[
    every node/.style={draw, circle, inner sep=0pt, minimum size=5pt},
    vertex/.style={fill},
]

% Define positions for the graphs
\foreach \i [evaluate=\i as \xi using {int((\i-1)*2+1)}] in {1,...,6}{
    % Define vertical positions for nodes in each graph
    \foreach \j in {1,...,9}{
        \node (g\i-\j) at (\xi,-1.5*\j) {};
    }
    % Label the graphs
    \node[above] at (\xi,0) {$\xi K_9$};
}

% Define vertices for the left labels
\foreach \i [evaluate=\i as \xi using {int(\i)}] in {1,...,9}{
    \node[vertex] (x\i) at (0,-1.5*\xi) {};
    \node[above] at (0,-1.5*\xi) {$x_\xi$};
}

% Define indices for the bottom labels
\foreach \i in {1,...,6}{
    \node at (\i*2-1,-1.6) {\i};
}

% Connect the vertices
\foreach \i in {1,...,6}{
    \foreach \j [evaluate=\j as \k using {int(mod(\j-1,3))}] in {1,...,9}{
        \ifnum\k=0
            \draw (g\i-\j) -- (g[\ifnum\i>1 int(\i-1)\else1\fi]-\the\numexpr\j+3);
        \fi
        \ifnum\k=1
            \draw (g\i-\j) -- (g[\ifnum\i>2 int(\i-2)\else1\fi]-\the\numexpr\j+6);
        \fi
        \ifnum\k=2
            \draw (g\i-\j) -- (g[\ifnum\i>3 int(\i-3)\else1\fi]-\the\numexpr\j+9);
        \fi
    }
}

\end{tikzpicture}
\end{document}