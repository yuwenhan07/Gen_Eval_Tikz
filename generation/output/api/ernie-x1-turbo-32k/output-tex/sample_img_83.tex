\documentclass[tikz,border=3mm]{standalone}
\usepackage{pgfplots}
\pgfplotsset{compat=1.18}

\begin{document}
\begin{tikzpicture}
\begin{axis}[
    ybar,
    enlargelimits=0.15,
    legend style={at={(0.5,-0.1)}, anchor=north, legend columns=-1},
    ylabel={$\hat{\tau}$},
    xlabel={Uniform Distributions},
    symbolic x coords={U5_100, U5_150, U5_200, U10_50, U10_80, U10_100},
    xticklabels={
        $\mathcal{U}^5([1,100])$,
        $\mathcal{U}^5([1,150])$,
        $\mathcal{U}^5([1,200])$,
        $\mathcal{U}^{10}([1,50])$,
        $\mathcal{U}^{10}([1,80])$,
        $\mathcal{U}^{10}([1,100])$
    },
    xtick=data,
    nodes near coords,
    nodes near coords align={vertical},
    cycle list/.define={my color}{
        {blue, fill=blue!30}, 
        {orange, fill=orange!30}
    },
    cycle list name=my color,
    bar width=0.6cm,
    width=12cm,
    height=7cm
]

% Dimension 5 data (hypothetical values)
\addplot coordinates {
    (U5_100, 0.42)
    (U5_150, 0.38)
    (U5_200, 0.35)
};

% Dimension 10 data (hypothetical values)
\addplot coordinates {
    (U10_50, 0.28)
    (U10_80, 0.31)
    (U10_100, 0.33)
};

\legend{Dimension 5, Dimension 10}
\end{axis}
\end{tikzpicture}
\end{document}