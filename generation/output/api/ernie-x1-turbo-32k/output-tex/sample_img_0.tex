\documentclass{standalone}
\usepackage{tikz}
\usetikzlibrary{arrows.meta, positioning, quotes}
\begin{document}
\begin{tikzpicture}[
    node distance=2cm,
    main node/.style={
        rectangle, 
        rounded corners, 
        draw=black, 
        thick, 
        minimum width=3cm,
        minimum height=1.5cm,
        align=center,
        font=\sffamily\bfseries
    },
    arrow/.style={
        -{Stealth[length=5pt]},
        thick,
        shorten >=2pt,
        shorten <=2pt
    }
]
    % Nodes
    \node[main node, fill=red!20] (Lagrangian) {Lagrangian Source\\ $S_L$};
    \node[main node, above right=of Lagrangian] (Eulerian) {Eulerian Source\\ $S_E$};
    \node[main node, below right=of Lagrangian] (Balance) {Balance Law\\ \eqref{EE}};
    \node[main node, above left=of Lagrangian, fill=blue!10] (Formulation1) {Formulation 1};
    \node[main node, below left=of Lagrangian, fill=blue!10] (FormulationN) {Formulation N};
    % Arrows with labels
    \draw[arrow, "determined by"'] (Eulerian) -- (Lagrangian);
    \draw[arrow, "defines"'] (Lagrangian) -- (Balance);
    \draw[arrow, "under Assumption \eqref{ass:h}"'] (Formulation1) -- (Lagrangian);
    \draw[arrow, "under Assumption \eqref{ass:h}"'] (FormulationN) -- (Lagrangian);
    \draw[arrow, dashed, "continuous"'] (Lagrangian) -- (Eulerian);
    
    % Annotations
    \node[above=0.5cm of Lagrangian, align=center, font=\itshape] {Non-degeneracy\\ Assumption \eqref{ass:h}};
    \node[below=1.5cm of Lagrangian, align=center, font=\small] {When $S_L$ is continuous,\\ it serves as the universal source term};
\end{tikzpicture}
\end{document}