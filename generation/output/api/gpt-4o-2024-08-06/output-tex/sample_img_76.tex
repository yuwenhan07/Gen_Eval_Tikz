\documentclass{standalone}
\usepackage{tikz}
\usepackage{amsmath}

\begin{document}

\begin{tikzpicture}

% First Diagram for {1^6, 7^18}
% Nodes
\foreach \i in {0,1,...,6} {
    \node[circle,fill,inner sep=1pt] (a\i) at (\i,0) {};
    \node[below] at (a\i) {\i};
}

\foreach \i in {7,8,...,13} {
    \node[circle,fill,inner sep=1pt] (a\i) at (\i-7,1) {};
    \node[above] at (a\i) {\i};
}

\foreach \i in {14,15,...,20} {
    \node[circle,fill,inner sep=1pt] (a\i) at (\i-14,2) {};
    \node[below] at (a\i) {\i};
}

\foreach \i in {21,22,...,24} {
    \node[circle,fill,inner sep=1pt] (a\i) at (\i-21,3) {};
    \node[above] at (a\i) {\i};
}

% Edges
\foreach \i in {0,1,...,5} {
    \draw[-] (a\i) -- (a\i+1);
}

\foreach \i in {7,8,...,12} {
    \draw[-] (a\i) -- (a\i+1);
}

\foreach \i in {14,15,...,19} {
    \draw[-] (a\i) -- (a\i+1);
}

\foreach \i in {21,22,...,23} {
    \draw[-] (a\i) -- (a\i+1);
}

\draw (a6) -- (a13);
\draw (a13) -- (a20);
\draw (a20) -- (a24);

% Second Diagram for {1^7, 7^{17}, 8^3}
% Nodes
\foreach \i in {0,1,...,6} {
    \node[circle,fill,inner sep=1pt] (b\i) at (\i+8,0) {};
    \node[below] at (b\i) {\i};
}

\foreach \i in {7,8,...,14} {
    \node[circle,fill,inner sep=1pt] (b\i) at (\i+1,1) {};
    \node[above] at (b\i) {\i};
}

\foreach \i in {15,16,...,21} {
    \node[circle,fill,inner sep=1pt] (b\i) at (\i-6,2) {};
    \node[below] at (b\i) {\i};
}

\foreach \i in {22,23,...,27} {
    \node[circle,fill,inner sep=1pt] (b\i) at (\i-15,3) {};
    \node[above] at (b\i) {\i};
}

% Edges
\foreach \i in {0,1,...,5} {
    \draw[-] (b\i) -- (b\i+1);
}

\foreach \i in {7,8,...,13} {
    \draw[-] (b\i) -- (b\i+1);
}

\foreach \i in {15,16,...,20} {
    \draw[-] (b\i) -- (b\i+1);
}

\foreach \i in {22,23,...,26} {
    \draw[-] (b\i) -- (b\i+1);
}

\draw (b6) -- (b14);
\draw (b14) -- (b21);
\draw (b21) -- (b27);

\end{tikzpicture}

\end{document}