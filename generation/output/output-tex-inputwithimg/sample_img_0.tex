\documentclass{article}
\usepackage{amsmath}
\usepackage{tikz}
\usetikzlibrary{shapes,calc,intersections}

\begin{document}

% Define the color for the circles
\tikzset{
    mycircle/.style={draw=#1!80, fill=#1!20},
}

% Define the circle names
\def\Borel{\text{Borel, bounded functions}}
\def\Eulerian{\text{Eulerian sources}}
\def\Lagrangian{\text{Lagrangian sources associated to a given $\chi$}}

% Draw the circles
\begin{tikzpicture}[scale=1.5]
  \fill[mycircle=cyan] (-4,-1) ellipse (4cm and 2cm);
  \fill[mycircle=magenta] (2,-1) ellipse (4cm and 2cm);
  \path [name path=A] (-4,-1) ellipse (4cm and 2cm);
  \path [name path=B] (2,-1) ellipse (4cm and 2cm);
  \fill [name intersections={of=A and B,by={x}}] (x) circle (3pt);
  \draw [thick, name path=C] (x) circle (2cm);
  \path [name intersections={of=C and A,by={x1}}];
  \path [name intersections={of=C and B,by={x2}}];
  \draw [name intersections={of=A and C},name path=D] (x1) -- (x2);
  \fill [name intersections={of=D and B,by={x3}}] (x3) circle (3pt);

  % Labels
  \node at (-4,-2.75) {\Eulerian};
  \node at (2,-2.75) {\Lagrangian};
  \node at (0,0) {\Borel};
  \node at (0,1) {$\S$ \ref{subsec:Borel}};
  \node at (0,0.25) {$\S$ \ref{subsec:Eulerian}};
  \node at (0,-0.25) {Broad sources};
\end{tikzpicture}

\end{document}