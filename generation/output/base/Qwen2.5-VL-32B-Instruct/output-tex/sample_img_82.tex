\documentclass[tikz,border=10pt]{standalone}
\usetikzlibrary{calc}

\begin{document}
\begin{tikzpicture}[scale=2, thick]

% Define the value of \epsilon_n for scaling
\def\en{1} % You can adjust this value as needed

% Define the vertices of the triangle
\coordinate (A) at (0, 0); % Point 0
\coordinate (B) at (2*\en, 0); % Point 1
\coordinate (C) at (\en, {\sqrt(3)*\en}); % Point 2

% Define the midpoint H of the base [01]
\coordinate (H) at (\en, 0);

% Define the centroid G (intersection of medians)
\coordinate (G) at ({(0 + 2*\en + \en)/3}, {(0 + 0 + {\sqrt(3)*\en})/3});

% Draw the triangle
\draw (A) -- (B) -- (C) -- cycle;

% Draw the medians
\draw[dashed] (A) -- (G);
\draw[dashed] (B) -- (G);
\draw[dashed] (C) -- (G);

% Draw vertical dashed line from C to H
\draw[dashed] (C) -- (H);

% Label the vertices
\node[circle, fill, inner sep=1.5pt, label=left:$0$] at (A) {};
\node[circle, fill, inner sep=1.5pt, label=right:$1$] at (B) {};
\node[circle, fill, inner sep=1.5pt, label=above:$2$] at (C) {};

% Label the centroid G
\node[circle, fill, inner sep=1.5pt, label=below right:$G$] at (G) {};

% Label the midpoint H
\node[circle, fill, inner sep=1.5pt, label=below:$H$] at (H) {};

% Label the regions or segments
\node at ($0.5*(A)!0.5!(G)$) {$C_1$};
\node at ($0.5*(B)!0.5!(G)$) {$C_2$};
\node at ($0.5*(C)!0.5!(H)$) {$C_3$};

\end{tikzpicture}
\end{document}