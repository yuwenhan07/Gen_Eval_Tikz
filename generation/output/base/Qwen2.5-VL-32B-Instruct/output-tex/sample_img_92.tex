\documentclass{standalone}
\usepackage{tikz}
\usetikzlibrary{arrows.meta}

\begin{document}
\begin{tikzpicture}[
  every node/.style={align=center},
  grow via three points={
    one child at (0.5,-0.7) and two children at (0.5,-0.7) and (1.0,-0.7)
  },
  edge from parent path={(\tikzparentnode.south) |- (\tikzchildnode.west)},
  level distance=1.2cm,
  sibling distance=1.5cm,
]

% Define the tree structure
\node {Summation Methods}
  child {node {Sequential}
    child {node {N\"orlund}
      child {node {Ces\'aro}}
    }
    child {node {Euler}
      edge from parent [draw=none]
    }
  }
  child {node {Functional}
    child {node {Abelian means}
      child {node {Abel}}
    }
    child {node {Lambert}
      edge from parent [draw=none]
    }
    child {node {Mittag-Leffler}
      child {node {weak Borel}
        child {node {strong Borel}}
      }
    }
  };

% Add directed edges for implications and special cases
\draw[-Latex] (Cesaro) -- node[above, pos=0.5] {$\longrightarrow$} (Abel);
\draw[-Latex] (Abel) -- node[above, pos=0.5] {$\twoheadrightarrow$} (WeakBorel);
\draw[-Latex] (WeakBorel) -- node[above, pos=0.5] {$\longrightarrow$} (StrongBorel);

% Name the nodes for easier reference
\node at (Cesaro.south west) [below left=3pt of Cesaro, anchor=east] {\textbf{Ces\'aro}};
\node at (Abel.south east) [below right=3pt of Abel, anchor=west] {\textbf{Abel}};
\node at (WeakBorel.south west) [below left=3pt of WeakBorel, anchor=east] {\textbf{weak Borel}};
\node at (StrongBorel.south east) [below right=3pt of StrongBorel, anchor=west] {\textbf{strong Borel}};

\end{tikzpicture}
\end{document}