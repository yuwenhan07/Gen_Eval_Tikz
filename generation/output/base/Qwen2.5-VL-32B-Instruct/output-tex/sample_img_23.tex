\documentclass{standalone}
\usepackage{tikz}
\usetikzlibrary{shapes,arrows.meta,positioning}

\begin{document}

\begin{tikzpicture}[node distance=2cm]

    % Define styles for the nodes
    \tikzset{
        vertex/.style={ellipse, draw, minimum width=1.5cm, minimum height=1.5cm, inner sep=0pt},
        dot/.style={circle, fill=black, inner sep=1.5pt},
    }

    % Draw the partite sets as ellipses
    \node[vertex] (V1) {$V_1$};
    \node[vertex, right=of V1] (V2) {$V_2$};
    \node[right=of V2] (dots) {$\cdots$};
    \node[vertex, right=of dots] (Vc_minus_1) {$V_{c-1}$};
    \node[vertex, right=of Vc_minus_1] (Vc) {$V_c$};

    % Add vertices inside each partite set
    \foreach \i in {1,...,3} {
        \node[dot, below=0.5cm of V1] (v1-\i) {};
    }
    \foreach \i in {1,...,3} {
        \node[dot, below=0.5cm of V2] (v2-\i) {};
    }
    \foreach \i in {1,...,3} {
        \node[dot, below=0.5cm of Vc_minus_1] (vc_minus_1-\i) {};
    }
    \node[dot, below=0.5cm of Vc] (vc-1) {};

    % Draw directed edges between partite sets
    \draw[-Stealth] (v1-1) -- (v2-1);
    \draw[-Stealth] (v2-1) -- ++(1.5,0) coordinate (mid1);
    \draw[dashed] (mid1) -- ++(1.5,0) coordinate (mid2);
    \draw[-Stealth] (mid2) -- (vc_minus_1-1);
    \draw[-Stealth] (vc_minus_1-1) -- (vc-1);

    % Draw the curved edge from V_c back to V_1
    \draw[-Stealth] (vc-1) to[out=-90,in=-90, looseness=2] (v1-1);

\end{tikzpicture}

\end{document}