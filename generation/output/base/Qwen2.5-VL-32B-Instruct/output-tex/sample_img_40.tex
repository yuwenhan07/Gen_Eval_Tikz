\documentclass[tikz,border=10pt]{standalone}
\usetikzlibrary{angles, quotes, intersections}

\begin{document}
\begin{tikzpicture}[scale=1.5]

% Define coordinates for the arcs and straight lines
\coordinate (O1) at (0,0); % Center of the first arc
\coordinate (O2) at (3,0); % Center of the second arc

% Draw the first large arc
\draw[name path=arc1] 
  (O1) ++(180:2.5cm) arc (180:60:2.5cm)
  node[midway, above] {\(\mu_1\)}
  -- ++(0:2.5cm) coordinate (A1);

% Draw the second large arc
\draw[name path=arc2] 
  (O2) ++(-60:2.5cm) arc (-60:-180:2.5cm)
  node[midway, below] {\(\mu_2\)}
  -- ++(0:2.5cm) coordinate (A2);

% Draw the straight lines connecting the arcs
\draw[name path=line1] 
  (O1) ++(180:2.5cm) -- ++(0:7cm) coordinate (B1);
\draw[name path=line2] 
  (O1) ++(60:2.5cm) -- ++(0:7cm) coordinate (B2);

% Mark the intersection points
\path [name intersections={of=arc1 and line1, by={I1}}];
\path [name intersections={of=arc2 and line2, by={I2}}];

% Label the intersection points
\node[circle, fill=black, inner sep=1pt, label=below:\(\zeta_1\)] at (I1) {};

% Draw the small gap caused by \tau_1
\draw[dashed] 
  (I1) ++(0:0.3cm) arc (0:60:0.3cm) coordinate (tau1);
\fill[white] (I1) circle (2pt); % Make the intersection point transparent

% Label the small gap
\node[above right] at (tau1) {\(\tau_1\)};

% Mark the angles
\pic[draw, angle radius=10mm, "$\alpha_1$", angle eccentricity=1.3] {angle = B1--O1--I1};
\pic[draw, angle radius=10mm, "$\beta_1$", angle eccentricity=1.3] {angle = I1--O1--B2};
\pic[draw, angle radius=10mm, "$\alpha_2$", angle eccentricity=1.3] {angle = O2--I2--B2};
\pic[draw, angle radius=10mm, "$\delta_1$", angle eccentricity=1.3] {angle = B2--O2--A1};

% Label the arcs and lines
\node[above left] at (O1) {\(\gamma_1\)};
\node[above right] at (O2) {\(\delta_1\)};
\node[below left] at ($(O1)!0.5!(B1)$) {\(\lambda_1\)};
\node[below right] at ($(O2)!0.5!(B2)$) {\(\lambda_2\)};
\node[below left] at (I1) {\(\gamma_2\)};
\node[right] at (A2) {\(\delta_2\)};

\end{tikzpicture}
\end{document}